\documentclass{article}\usepackage{graphicx}
\usepackage[a4paper, margin=1in]{geometry}
\usepackage{hyperref}
\usepackage{titlesec}
\usepackage{todonotes}

\titleformat{\section}{\large\bfseries}{}{0em}{}

\title{\textbf{Chat Vibe Check: Project Proposal}}
\begin{document}
\maketitle

\section{Project Metadata}
\begin{itemize}
    \item \textbf{Project Title:} Chat Vibe Check
    \item \textbf{Team Members Name:} 
    \begin{itemize}
        \item Deepa Rukmini Mahalingappa (34040788, drukminimaha@umass.edu)
        \item Benjamin Ibasco (28561432, bibasco@umass.edu)
        \item Javin Mendiratta (33328658, jmendiratta@umass.edu)
    \end{itemize}
   
    \item \textbf{Project Repository:} \href{https://github.com/bibascoumass/chat-vibe-check}{GitHub Repository Link}
\end{itemize}

\section*{Overview and Motivation}
Provide an overview of the project goals and the motivation for it. Consider that this will be read by people who did not see your project proposal.

\section*{Related Work}
The main inspiration of the project mainly came from the fact that there is growing empirical evidence that arranged marriages are scoring better on metrics like satisfaction and divorce rates compared to traditional marriages (\href{https://www.psychologytoday.com/us/blog/the-science-behind-behavior/201511/why-are-so-many-indian-arranged-marriages-successful}{PyschologyToday.com article}). At the time, we were interested in exploring the reasons why and quickly realized this would be a challenging goal . We then explored narrowing the scope but staying on the theme of understanding relationships and was inspired by personal chat visualizations done by users analyzing trends in their relationship (\href{https://github.com/bibascoumass/chat-vibe-check}{Example from Reddit}). From the examples we noticed the visualizations or insights were very surface level and thought that there might be use for a tool that tried to explore more meaningful behavioral patterns from the chat data. To get a better direction on how or what we would try to observe from the chat data, we decided to base our analysis around an actual psychological framework and we chose Attachment Theory for this.  

\section*{Questions}
The main question we're trying to answer is what behavioral patterns can we observe from chat message interactions between people two get a better understanding of their attachment styles? As we explored this idea, we focused on two key derived attributes: Sentiment and engagement. From this realization we focused on having our visualizations answer the following questions:
\begin{itemize}
    \item How does the sentiment of my conversation evolve over time?
    \item Are there certain conversation topics that are intensly positve or negative? 
    \item Are there periods of time or topics in my conversation when a person is less responsive than usual or tries to avoid or change the topic? 
\end{itemize}

\section*{Data}
Include information about the source, how you collected it (e.g., web scraping), cleaning methods, etc.
\begin{itemize}
    \item Ubuntu chat logs: https://www.kaggle.com/datasets/rtatman/ubuntu-dialogue-corpus/data
    \item Movie dialogue: \texttt{https://www.kaggle.com/datasets/Cornell-University/movie-dialog-corpus/data}
    \item WhatsApp Chat used for studying cyberbullying in Italy: \texttt{https://github.com/dhfbk/WhatsApp-Dataset}
    \item WhatsApp Group Chat in Hindi: \texttt{https://www.kaggle.com/datasets/mmuhammetcavus/whatsapp-chat/data}
\end{itemize}

\section*{Exploratory Data Analysis}
What visualizations did you use to initially look at your data? What insights did you gain? How did these insights inform your design?
\begin{itemize}
    \item \textbf{Download:} We'll need to make sure that our website references instructions on how to download the data from our supported platforms.
    \item \textbf{Cleanup:} We'll need to ingest data from multiple formats and store it such that we can at least query for the message itself and the following metadata: sender, receiver, send time, seen time (this may not be available on all platforms).
    \item \textbf{Quantities:} The quantities we expect to derive from the data are sentiment, response time, engagement, and conflict resolution metrics relative to topics and time.
    \item \textbf{Analysis:} We've decided for now that we'll keep the scope simple in the beginning and only focus on text messages and emojis. Video, audio, GIFs, and images will be ignored in the onset.
\end{itemize}

\section*{Design Evolution}
What are the different visualizations you considered? Justify the design decisions you made using the perceptual and design principles you learned in the course. Did you deviate from your proposal?

\section*{Implementation}
Describe the intent and functionality of the interactive visualizations you implemented. Provide clear and well-referenced images showing the key design and interaction elements.
\begin{itemize}
    \item Time (X-axis) vs Sentiment (Y-axis) Scatterplot: Each point represents a message and each point is labeled by user. 
    \item Topic (X-axis) vs Sentiment (Y-axis) Heatmap: Each cell represents the average sentiment score for messages the belong to that topic. 
    \item Response Time vs Sentiment (Y-axis) Scatter Plot: Each point represents a message and each point is labeled by user. 
    \item Conversation Display: Clicking on a message plot point displays a modal overlay that shows the entire conversation in a scrolling view like a messaging app would.
\end{itemize}

\section*{Optional Features}
\begin{itemize}
    \item Message Grouping: We’ll need to decide if we’ll try to group messages together as part of the same conversation.
        \begin{itemize}
            \item We can base this off two criteria: topic and time. Starting with time-based grouping and then later by topic if time permitting.
        \end{itemize}
    \item Conflict Resolution Snakey Diagram: Depicting the flow of conversation state transitions from neutral to conflict to resolution.
    \item Give an attachment style assessment with a confidence score.
    \item Show each user's perceived attachment style over time.
    \item Show patterns of behavior based on sentiment and attachment style analysis relative to time and topics.
\end{itemize}
(Note: We’ve decided to make visualizations that try to use a derived attachment style metric optional, as we currently have no guarantee if the derived attributes conclusively determine attachment style.)

\section*{Security}
\begin{itemize}
    \item Client-side processing such that our data consists only of the messages, the timestamps and a generic username assigned like "User A" or "User B".
    \item Use our chosen hosting platform's (like Heroku) built-in HTTPS.
    \item Then we always want to delete the data once the user who has uploaded ends their session, so we're planning on using an in-memory datastore (like Redis) configured with TLS and disable any of its features that write the data to disk (like backups). This way we avoid needing to encrypt data on disk.
\end{itemize}

\section*{Tech Stack}
\begin{itemize}
    \item Front End: React, D3, Modal overlay library (react-modal?)
    \item Back End: Node.js + Express.js (Express for data ingestion REST APIs), Redis, some NLP library
    \item Deployment: Heroku, GitHub
\end{itemize}

\section*{Evaluation}
What did you learn about the data by using your visualizations? How did you answer your research questions? How well does your visualization work, and how could you further improve it? (Be honest here. Limitations are a part of any project, and they will be noticeable during the grading process. Acknowledging them in this section indicates thoughtfulness in your design process, and, as such, will only help your grade.)


\end{document}