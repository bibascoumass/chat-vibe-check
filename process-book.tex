\documentclass{article}\usepackage{graphicx}
\usepackage[a4paper, margin=1in]{geometry}
\usepackage{hyperref}
\usepackage{titlesec}
\usepackage{todonotes}

\titleformat{\section}{\large\bfseries}{}{0em}{}

\title{\textbf{Chat Vibe Check: Project Proposal}}
\begin{document}
\maketitle

\section{Project Metadata}
\begin{itemize}
    \item \textbf{Project Title:} Chat Vibe Check
    \item \textbf{Team Members Name:} 
    \begin{itemize}
        \item Deepa Rukmini Mahalingappa (34040788, drukminimaha@umass.edu)
        \item Benjamin Ibasco (28561432, bibasco@umass.edu)
        \item Javin Mendiratta (33328658, jmendiratta@umass.edu)
    \end{itemize}
   
    \item \textbf{Project Repository:} \href{https://github.com/bibascoumass/chat-vibe-check}{GitHub Repository Link}
\end{itemize}

\section*{Overview and Motivation}
Provide an overview of the project goals and the motivation for it. Consider that this will be read by people who did not see your project proposal.

\section*{Related Work}
The main inspiration of the project mainly came from the fact that there is growing empirical evidence that arranged marriages are scoring better on metrics like satisfaction and divorce rates compared to traditional marriages (\href{https://www.psychologytoday.com/us/blog/the-science-behind-behavior/201511/why-are-so-many-indian-arranged-marriages-successful}{PyschologyToday.com article}). At the time, we were interested in exploring the reasons why and quickly realized this would be a challenging goal . We then explored narrowing the scope but staying on the theme of understanding relationships and was inspired by personal chat visualizations done by users analyzing trends in their relationship (\href{https://github.com/bibascoumass/chat-vibe-check}{Example from Reddit}). From the examples we noticed the visualizations or insights were very surface level and thought that there might be use for a tool that tried to explore more meaningful behavioral patterns from the chat data. To get a better direction on how or what we would try to observe from the chat data, we decided to base our analysis around an actual psychological framework and we chose Attachment Theory for this.  

\section*{Questions}
The main question we're trying to answer is what behavioral patterns can we observe from chat message interactions between people two get a better understanding of their attachment styles? As we explored this idea, we focused on two key derived attributes: Sentiment and engagement. From this realization we focused on having our visualizations answer the following questions:
\begin{itemize}
    \item How does the sentiment of my conversation evolve over time?
    \item Are there certain conversation topics that are intensly positve or negative? 
    \item Are there periods of time or topics in my conversation when a person is less responsive than usual or tries to avoid or change the topic? 
\end{itemize}

\section*{Data}
To prototype the data pre-processing and ingestion implementations, we decided to just work with our own Whatsapp group chat data as we just needed to work with the data format for the messaging platform we wanted to support. 
\\\\ To prototype the visualizations, it was difficult to find a WhatsApp chat dataset in English so we settled on using a dataset from Kaggle of over one million two-person chat logs related to Ubuntu technical support: https://www.kaggle.com/datasets/rtatman/ubuntu-dialogue-corpus/data . At this early stage, there hasn't been a need to do any pre-processing on this dataset. 
\\\\ The following datasets were also considered:
\begin{itemize}
    \item \href{https://github.com/dhfbk/WhatsApp-Dataset}{Movie Dialogue Corpus} - Initially chosen as a backup in the worst case scenario wherein we couldn't find two person conversation chat data in English. 
    \item The following WhatsApp chat datasets were considered and although there are analysis libararies that support multiple languages (e.g. multilang-sentiment) we ultimately avoided these as we wouldn't be able to compare the results of our analysis to what we thought the data was actually expressing. 
    \begin{itemize}
        \item \href{https://www.kaggle.com/datasets/Cornell-University/movie-dialog-corpus/data}{WhatsApp Chat used for studying cyberbullying in Italy}
        \item \href{https://www.kaggle.com/datasets/mmuhammetcavus/whatsapp-chat/data}{WhatsApp Group Chat in Hindi}
    \end{itemize}
\end{itemize}

\section*{Exploratory Data Analysis}
What visualizations did you use to initially look at your data? What insights did you gain? How did these insights inform your design?

\section*{Design Evolution}
The project aimed to extract and visualize emotional patterns from chat data in a way that is easy to interpret, interactive, and actionable. Our design evolved significantly from the initial proposal as we explored more effective ways to represent time-based and user-based sentiment trends.

\begin{itemize}
    \item \textbf{Initial Visualizations Considered:}
    \begin{itemize}
        \item Bar charts to show average sentiment per user or per day.
        \item Line charts for sentiment flow over time.
        \item Pie charts to show distribution of sentiment types.
        \item Snakey Diagram showing the flow of conversation state transitions from neutral to conflict to resolution.
    \end{itemize}

    \item \textbf{Final Visualizations Used:}
    \begin{itemize}
        \item \textbf{Heatmap:} Showcased sentiment variation across time periods (e.g., hourly/daily), helping identify emotional peaks.
        \item \textbf{Scatter Plot - Response Time vs Sentiment:} Revealed correlation between emotional intensity and reply latency.
        \item \textbf{Scatter Plot - Timestamp vs Sentiment:} Illustrated sentiment flow throughout the conversation, supporting detection of escalating or resolving emotional trends.
        \item \textbf{Chat Parser Table:} Provided a clean, searchable layout of individual messages with timestamps, user labels, and sentiment tags.
    \end{itemize}    

    \item \textbf{Deviations from Original Proposal:}
    \begin{itemize}
        \item Initially proposed line charts were replaced by scatter plots, which better captured the irregular and message-level nature of chat data.
        \item We're currently playing around relating time, topic and sentiment instead all in one chart rather than having a two separate ones for topic and then for time.
        \item The most basic implementation for showing the chat is currently just having it as a separate page but we will look into the displaying it as the user clicks on messages. 
    \end{itemize}
\end{itemize}
 \section*{Visualizations Used}

\begin{itemize}
    \item \textbf{Heatmap: Sentiment vs Time}
    \begin{itemize}
        \item \textbf{Design Goal:} To identify how sentiment changes over time (across hours or days), highlighting peak emotional periods like high negativity during specific hours.
        \item \textbf{Color Mapping:}
        \begin{itemize}
            \item \textcolor{red}{Red} = Negative
            \item \textcolor{green}{Green} = Positive
            \item \textcolor{gray}{Gray} = Neutral
        \end{itemize}
    \end{itemize}
    
    \item \textbf{Scatter Plot: Response Time vs Sentiment}
    \begin{itemize}
        \item \textbf{Design Goal:} To explore whether the speed of replies correlates with emotional tone, revealing patterns like faster responses during emotional conversations.
        \item \textbf{Design Enhancements:}
        \begin{itemize}
            \item Color-coded sentiment points for clarity.
            \item Tooltips on hover for detailed insights.
            \item Filter options by user or time range.
        \end{itemize}
        \item \textbf{Design Principles Used:}
        \begin{itemize}
            \item Data-ink ratio for clean visualization.
            \item Pre-attentive processing with color and shape to distinguish sentiment.
        \end{itemize}
    \end{itemize}
    
    \item \textbf{Scatter Plot: Timestamp vs Sentiment}
    \begin{itemize}
        \item \textbf{Design Goal:} To display sentiment flow chronologically over the chat session, allowing trend detection like escalating negativity or emotional recovery.
        \item \textbf{Design Decisions:}
        \begin{itemize}
            \item Smooth timeline across X-axis.
            \item Hover details with sentiment classification.
        \end{itemize}
        \item \textbf{Design Principle Used:}
        \begin{itemize}
            \item Temporal alignment of visual elements for easy scanning.
            \item Consistent color scale to maintain cognitive ease.
        \end{itemize}
    \end{itemize}
    
    \item \textbf{Chat Parser with Sentiment Tags}
    \begin{itemize}
        \item \textbf{Design Goal:} To convert raw chat into a clean, readable, and searchable table with labeled sentiment.
        \item \textbf{Features:}
        \begin{itemize}
            \item Each message row includes: Timestamp, User, Message, and Sentiment Score (with tag).
            \item Sentiment tags are color-coded.
            \item Optional search/filter by user or sentiment type.
        \end{itemize}
        \item \textbf{Design Principles Used:}
        \begin{itemize}
            \item Clarity and hierarchy through formatting, spacing, and alignment.
            \item Redundancy principle — sentiment shown via both text and color for accessibility.
        \end{itemize}

        
    \end{itemize}
\end{itemize}



\section*{Implementation}
Describe the intent and functionality of the interactive visualizations you implemented. Provide clear and well-referenced images showing the key design and interaction elements.
\begin{itemize}
    \item Time (X-axis) vs Sentiment (Y-axis) Scatterplot: Each point represents a message and each point is labeled by user. 
    \item Topic (X-axis) vs Sentiment (Y-axis) Heatmap: Each cell represents the average sentiment score for messages the belong to that topic. 
    \item Response Time vs Sentiment (Y-axis) Scatter Plot: Each point represents a message and each point is labeled by user. 
    \item Conversation Display: Clicking on a message plot point displays a modal overlay that shows the entire conversation in a scrolling view like a messaging app would.
\end{itemize}

\section*{Optional Features}
\begin{itemize}
    \item Message Grouping: We’ll need to decide if we’ll try to group messages together as part of the same conversation.
        \begin{itemize}
            \item We can base this off two criteria: topic and time. Starting with time-based grouping and then later by topic if time permitting.
        \end{itemize}
    \item Conflict Resolution Snakey Diagram: Depicting the flow of conversation state transitions from neutral to conflict to resolution.
    \item Give an attachment style assessment with a confidence score.
    \item Show each user's perceived attachment style over time.
    \item Show patterns of behavior based on sentiment and attachment style analysis relative to time and topics.
\end{itemize}
(Note: We’ve decided to make visualizations that try to use a derived attachment style metric optional, as we currently have no guarantee if the derived attributes conclusively determine attachment style.)

\section*{Security}
\begin{itemize}
    \item Client-side processing such that our data consists only of the messages, the timestamps and a generic username assigned like "User A" or "User B".
    \item Use our chosen hosting platform's (like Heroku) built-in HTTPS.
    \item Then we always want to delete the data once the user who has uploaded ends their session, so we're planning on using an in-memory datastore (like Redis) configured with TLS and disable any of its features that write the data to disk (like backups). This way we avoid needing to encrypt data on disk.
\end{itemize}

\section*{Tech Stack}
\begin{itemize}
    \item Front End: React, D3, Modal overlay library (react-modal?)
    \item Back End: Node.js + Express.js (Express for data ingestion REST APIs), Redis, some NLP library
    \item Deployment: Heroku, GitHub
\end{itemize}

\section*{Evaluation}
\begin{itemize}
    \item For such a project that asks users to require them to upload potentially sensitive data, the tool for sanitizing and removing that data needs to be within their control and trust. This was the main reason why we weren't able to initially get data from real people within our networks. 
    \item We still need to fix how we're identifying topics as the current implementation is finding it hard to identify specific topics in the content and just labeling a lot of data as "Other"
    \item Admittedly, we may have spent too much time expirmenting with applying the visualizations even in group settings and trying to group messages by topic/conversation so we still need to finetune the scatter plot visualization as well in terms of scoping the data to specific conversations to have more meaningful insights. 
\end{itemize}


\end{document}
